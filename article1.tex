\title{Rethinking and designing lifelong learning policy}
\subtitle{A short essay}
%\shorttitle{Do nothing improve the fat layer}
\author{Iris Merkelbach
\thanks{Iris Merkelbach is currently pursuing a second Master’s degree in Public Policy and Human Development, while simultaneously completing her Master’s degree in European Studies on Society, Science and Technology. Her main areas of interests are regional governance and the societal impact of technology and science. Corresponding author: \href{mailto:i.merkelbach@outlook.com}{i.merkelbach@outlook.com}.}  
}
\shortauthor{I. Merkelbach,}
%\institution{MPP Alummni, 2017-2018}
\date{\today}
\maketitle
\begin{abstract}
    To address human participation in an era of digitalization and AI, policy makers need to redesign their concept of lifelong learning. However, to put the concept lifelong learning successfully into practice, three policies should be implemented: the school curriculum needs to include informal community-based learning activities, policies targeting the private sector should enhance flexibility for formal educational and vocational learning activities, and lifelong learning needs to be funded by means of a combined private-public saving scheme.
\end{abstract}
\begin{keywords}
    Lifelong learning, digitalization, AI, employment, non-employment
\end{keywords}

\section{Introduction}
%\begin{figure}[b!]
%\includegraphics[width=\linewidth]{example-image-a}
%\caption{The A of hibernAtion}
%\end{figure}
The concept of lifelong learning has drawn to the attention of scholars and policy makers for decades. However, up until recent technological advancements, there was no incentive to pursue lifelong learning. Digitalization and the development of artificial intelligence (AI) have provided a reason for urging policy makers to turn the idea of lifelong learning into practice. According to Frey and Osborne, approximately 47 percent of total US employment is exposed to automation (2013). Contrary to this high estimate, another group of scholars casted a modest prediction model for future employment, estimating 18.2 percent of jobs to be at high risk of automation (\citealt{arntz2017revisiting}). Which forecasting model might eventually turn out to be society’s reality remains to be seen, but the undeniable fast pace of technological development begs for an answer on looking within and beyond the labor market. Lifelong learning could be one of the answers addressing human participation in the age of digitalization and AI, but how do policy makers motivate individuals and ensure sufficient funding ? Both issues, which will be discussed here, need to be considered, as they prevent the implementation of lifelong learning. 

\section*{The concept of lifelong learning}

Before turning to the stimulation of individual participation and creation of funding models for lifelong learning, the definition of lifelong learning needs to be made clear first. Lifelong learning has been described by the British Economic and Social Research Council as: 

‘’\textit{in which all citizens acquire a high quality general education, appropriate vocational training and a job ... while continuing to participate in education and training throughout their lives.... Citizens of a learning society would ... be able to engage in critical dialogue and action to improve the quality of life of the whole community.}’’ (Gorard and Rees, 2002)

Policy makers define lifelong learning not merely in terms of meeting skill-relevance demands of the job market, their definition relates to good citizenship and being a valuable member to society as well. Lifelong learning thus encompasses economic and social value. Since newly acquired skills serve a broader purpose, lifelong learning can be tied to non-economic and voluntary projects. Given the forecasting models mentioned earlier on, a significant part of the labor force will be replaced. While job replacement by automation is considered part of the economic growth cycle, it is predicted that digitalization and AI will have a permanent effect on job replacement. Among several reasons, current technological development causes stagnant wages and divergence between productivity growth and wage growth (Virgillito, 2016). As such, lifelong learning having a broader purpose, serves the interest of both employed and non-employed population. However, the definition falls short on the informal aspect of learning or, phrased precisely, learning opportunities related to community- based activities (Gorard and Rees, 2002) . Engaging permanently non-employed citizens regularly throughout their lives in formal educational forms such university or vocational is highly unlikely. Therefore, the definition of what comprises lifelong learning should be revised in order to incorporate the scenario of non-employability in an era of digitalization and AI:
‘’\textit{having all citizens involved in either informal community-based learning activities or formal educational and vocational learning activities throughout their lives in order to be a productive member of society and/or a valuable participant in the economy.}’’

Within the essay, the revised definition of lifelong learning will be employed, as a way to refer to both employed and non-employed citizens.

\section*{Stimulating individual participation in lifelong learning}

The inclusion of both informal and formal learning activities require different policy strategies. So far, educational policies have been focusing on the preparation for a career, while neglecting community-based learning activities. As a result, positive experiences of individuals at school are limited to career opportunities. Based on current numbers, unpaid work is already an important factor in the economy. Unpaid work for women in the developing as well as the industrial world is estimated to constitute two-third of their total work time, whereas it accounts for approximately one-third of total work time for men. Technological advancements will most likely alter the balance between paid and unpaid work (ILO, 2017; Swiebel, 1999). Although some policy makers have been concerned with citizens helping themselves as a way of mobilizing civil society (Usher and Edwards, 2007), prospects of employability remain important in the debate of lifelong learning (Hyde and Phillipson, 2014). A shift in mindset is needed, as educational policies should include informal learning opportunities into their curriculum. If a significant portion of the population will permanently remain unemployed, the positive experience of community-learning activities starting at school will become just as relevant as the preparation for and continuation of a career in the light of lifelong learning. Positive informal learning experiences will stimulate individuals to develop their capabilities outside a work-context. 
For formal learning activities, there exist multiple options, but effective participation of employed individuals in formal lifelong learning depends on the organizational structure of a company. Top-down organizations or organizations that insufficiently take into account work and personal needs of employees tend to see employees less motivated to participate in lifelong learning (Keeling, Jones, Botterill, and Gray, 2006). Of equal importance are flexible learning policies and learning policies incorporating long-term leave or near retirement (ibid). Policy makers should therefore design policies aimed at the private sector which increase the adoption of a flexible approach among companies towards lifelong learning. For instance, a policy measure could include a paid maximum of three weeks devoted to formal learning each year. Ensuring income during formal learning opportunities could function as an incentive for lifelong learning participation among employees.

\section*{Funding lifelong learning}

To support informal and formal lifelong learning, a saving scheme should be adopted by government, as regular investments necessary for lifelong learning cannot be merely funded by subsidies. A lesson can be drawn in this respect from pension schemes. Many OECD countries employ a combination of private and public funding. In these countries, private-pension contributions have been made mandatory for employed individuals as a means of supplementing their public pension scheme (\citealt{antolin2009filling}). The advantage of such a system is that it decreases public pension entitlement rates by requiring individuals to provide partially for their own retirement income. At the same time, as part of the retirement savings operate on a public scheme, individuals do not bear the entire risk of saving for old age (Queisser, Whitehouse and Whiteford, 2007). A public saving scheme for lifelong learning faces two challenges. First, public pension schemes have already been under pressure caused by the unsustainability of the public fund’s saving mechanism, which means that there might be a lack of willingness to create an additional saving scheme for lifelong learning. Second, a significant share of the working population is threatened to be replaced by digitalization and AI, which will put an additional pressure on public scheme mechanisms for lifelong learning. Nevertheless, a combined public-private funding scheme for lifelong learning would be a reasonable solution to address both concerns, given the perks explained earlier on. Public resistance primarily depends on whether the financial burden is perceived to be properly distributed among the employed population vis-à-vis individuals carrying the risk. Careful justification and explanation is vital in communicating the saving scheme to the public. Perceived fairness though, might conflict with the second concern. In terms of the future scenario of the workforce and the fund’s sustainability, policy makers most likely need to tilt the combined public-private scheme towards private savings. A possible negative effect of a combined public-private saving scheme tilled towards private savings would be that it could cause inequality between employed and non-employed individuals, as non-employed individuals cannot rely on private funding schemes (Gruber, 2013). As such, policy makers will need to make a trade-off between publicly perceived fair proportionality and sustainability of the saving scheme. 

\section*{Conclusion}

If the presented policy suggestions and the funding model would be incorporated in the policy design, lifelong learning could turn out to be an effective tool for addressing human participation in the era of digitalization and AI. However, crucial for successful implementation is the definition of lifelong learning, which is not merely a matter of policy redesign, but also requires a societal shift of attitude towards unpaid work. Academics and policy makers are front runners in this regard, and should prepare society for a future of lifelong learning.  

\bibliography{References}        %use a bibtex bibliography file refs.bib
\bibliographystyle{apacite} 