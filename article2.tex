\title{Comparative Analysis of the trade performance of the Southern African Development Community \& the Pacific Alliance}
%\subtitle{A short essay}
%\shorttitle{Do nothing improve the fat layer}
\author{Tim Höflinger
\thanks{Tim Höflinger is a graduate student at UNU-Merit \& Maastricht University with past work-experiences at the German Council on Foreign Relations and the Singaporean-German Chamber of Commerce. Currently he is placed as an intern at the United Nations Office on Drugs and Crime (UNODC) in Yangon, Myanmar. . Corresponding author: \href{mailto:t.hoflinger@student.maastrichtuniversity.nl}{t.hoflinger@student.maastrichtuniversity.nl}. All remaining
	errors are from the author. Final revision accepted: Sep 22, 2018. \footnotesize{Maastricht Policy Journal\textsuperscript{\textcopyright}} }  
}
\shortauthor{T. hoflinger,}

%\institution{MPP Alummni, 2017-2018}

\maketitle

\begin{abstract}
   This article assesses and compares the performance of the Pacific Alliance and the SADC in trade. The analytical framework encompasses four performance criteria respectively evaluated through intra-regional trade shares, FDI, trade costs and institutional design. The analysis found that the SADC is effective in fostering intra-regional trade but has a legitimacy problem primarily in regard to its donor dependence. The PA lacks effectiveness in increasing intra-regional trade, but is successful in promoting itself as a trade-bloc to the external-market and attracting FDI. The SADC’s FDI-inflow appears to be highly volatile to commodity price fluctuations. In terms of trade-efficiency, time and monetary trade barriers in the SADC are significantly higher compared to the PA, signalizing a lack of efficiency in decreasing these input costs in relation to the output. One ramification of these findings is that the SADC is effective in trade integration but at too high costs and that the PA is more focused on extra-regional trade as a solidified bloc.
\end{abstract}
\begin{keywords}
    SADC, Pacific Alliance, trade, intra-regional trade, FDI, trade costs
\end{keywords}

\section{Introduction}

In the following sections, this paper will assess and compare the performance of the Southern African Development Community (SADC) and the Pacific Alliance (PA) in the domain of trade. The subsequent section will elaborate on the analytical framework, followed by the main analysis, which results and ramifications will finally be discussed in a critical summary.

Initially, it is useful to outline some general differences and similarities of the two regional integration organizations (RIO). While the SADC’s establishment dates back to 1992, the PA is a relatively new organization founded in 2011. The alliance was created upon already existing trade agreements among the four states, as a supplementation-tool. The SADC on the other hand, emerged out of the SADCC, which was established in 1980 in order to construct a counterpoint to apartheid South Africa (\citealt{penfold2017regionalism}). The SADC, as the largest RIO in South Africa, comprises 15 South African member states, while the PA’s four members are spread across North- and South-America. The most notable dissimilarity however, is the divergence in regional economic conditions and output. The SADC accumulated a total GDP of 630.1 billion US\$ in 2015, in the same year the smaller PA generated a total GDP of 1.9 trillion US\$. 

Conversely, there are also resemblances between the two RIO’s such as in population size (PA: 224 million; SADC: 277 million) and the economic growth rates (PA: 2.6\%; SADC: 2.2\% (2015)). The organizational objective-setting of both RIOs for trade policy are quite typical. The PA seeks to “Build….an area of deep economic integration” and “Promote the growth, development and competitiveness of the Parties’ economies, aiming at achieving greater welfare…”.  Similarly, the SADCs objectives include aspects like “Achieve development and economic growth, alleviate poverty…” and “Promote self-sustaining development on the basis of collective self-reliance…”.  
   
The PA’s key competencies in trade are trade liberalization and the coordination of trade promotion activities. The key accomplishment of the PA is the in 2013 agreed on eradication of 92\% of the tariffs on intra-regional goods trade and the commitment to total elimination by 2020. A further noticeable achievement is the integration of the member states stock exchanges to a Latin-American Integrated Market.  The PA states are outperforming other Latin-American countries in most economic indicators such as unemployment rate, GDP growth and level of investment.  This fact, the integration pace and the outwards-orientated focus of the Alliance, might explain the high number of observer states that is has attracted. The mandate of the SADC in the domain of trade is the financial liberalization and coordination of regional trade. The most noteworthy achievements of the SADC in the trade domain are the formation of a free trade area in 2008 and the maximum tariff liberalization in 2012. A recent increase in Foreign Direct Investment (FDI) and intra-regional trade was attributed to these efforts of the RIO. 

\section{Methodology}

In order to analyze the performance of the selected regional organizations, I use Jörgensen’s   performance-criteria: relevance, effectiveness, efficiency/financial viability and additionally legitimacy/accountability. Relevance is measured in the amount of FDI-inflow that the respective region has attracted. This indicator reflects the valuation by extra-regional stakeholders, in particular companies, for a region in terms of investment and trade potential.  An increased FDI-inflow can indicate the recognition and support by the international market of a regional organization as a relevant and functioning economic and trade platform. The Data was taken from UNCTAD. As a limitation, it must be noted that the FDI indicator only takes into account equity flows.
Effectiveness is measured in intra-regional trade shares in export and import. This indicator provides a good insight in the achievement of the typical objectives formulated by regional organizations when it comes to economic integration such as an increased circulation of intra-regional goods, less trade barriers and an increased economic interdependence. Since this indicator is captured as a percentage of the total import/exports of the region, it takes into account the total trade-volume with the world as a control variable making the two RIOs with different economic development-levels more comparable. A limitation of this indicator is the availability of accurate trade data, in particular when it comes to the SADC. The indicator values for the PA were self-calculated by using raw-data from the UN-COMTRADE database (Appendix-1). The values for the SADC were taken from the official SADC Statistics Yearbook.  

For the assessment of organizational legitimacy and accountability, this paper will examine the respective institutional designs descriptively. Hereby, the focus is primarily on the identification and evaluation of input/control legitimacy in the form of dispute-settlement bodies, audit units, donor dependence and citizen representation/parliamentarianism. These input legitimacy components are adopted from van der Vleuten’s analytic framework. 
    
For financial viability/efficiency, the paper looks at the objective barriers of trade (trade costs) and its relation to total regional trade (output). Thereby the focus is on time and costs accompanying the export of goods as determinants of trade-efficiency, components which are interlinked with trade-integration performance overall. The Data is taken from the World Bank’s DOING BUSINESS project. The analytical focus is here on whether the RIO was able to improve efficiency by reducing trade costs on the regional trade process level. A limitation is the non-inclusion of further trade resistance factors such as population and distance (requires gravity-model estimation).

\section{Conclusions}

If one is required to strictly sum up the performance according to the discussed criteria I would conclude the following points: The SADC is comparatively effective in fostering intra-regional trade but has a legitimacy problem primarily in regard to its donor-dependence. Such a problem cannot be identified in the PA, but the Alliance lacks effectiveness in increasing intra-regional trade. The SADCs external-relevance in terms of FDI is on the gradual rise but from an instable nature due to the high dependency on commodities. While the PA is successful in promoting itself as a trade bloc to the external-market, attracting FDI and attention in the form of observer states. In terms of trade- efficiency, time and monetary trade barriers in the SADC are significantly higher compared to the PA, signalizing a lack of efficiency in decreasing these input costs in relation to the output.

But clearly such a rather rigorous performance judgment does not take into account the full context and all the influencing factors. Although measuring certain outcomes against certain standards and objectives is helpful to capture progress, one can mostly not make an objective and definitive judgment. The reality is more nuanced and contextual, as performance is a dependent variable impacted by numerous known and hidden factors. For example, it is possible that the SADC member economies would have behaved similarly even without the establishment of the RIO but simply as the effect of their demographic development and globalization. 

Moreover, to which degree can one really refer to better performance after a comparative-analysis, considering e.g. the large difference in economic development of the SADC and the PA.? What is considered weak in terms of progress in the PA might be considered strong in the SADC and vice versa. Just because the PA is not boosting intra-regional trade one cannot argue that this RIO is failing, as in the regional context intra-regional trade is indeed above the LA level. This also reveals the weakness of the benchmarking against global standards, as the world is not characterized by one but a multitude of varying levels of development velocities.

RIO performance is not static, like the organizations activities it constantly operates in a dynamic environment interacting with several factors. Looking only at certain numbers and selecting certain cases can never fully reduce the bias and endogeneity, implying that such a performance assessment will always be subjective. However, it is de facto the only way to form some kind of opinion about the progress and usefulness of RIOs. At least it can capture certain progress tendencies and insights of organizational activities, helpful to derive policy recommendations.


\bibliography{References}        %use a bibtex bibliography file refs.bib
\bibliographystyle{apacite}
\copyrightnotice
